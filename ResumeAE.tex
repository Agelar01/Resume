\documentclass{article}
\usepackage[utf8]{inputenc}
\usepackage[paper=a4paper, left=2cm, right=2cm, bottom=2cm, top=2cm]{geometry}
\usepackage{fontawesome}
\usepackage{hyperref}
\usepackage{multicol}


\usepackage{xcolor}
\definecolor{darkred}{rgb}{0.55, 0.0, 0.0}

\hypersetup{
    colorlinks = true,
    breaklinks = true
    linkcolor  = darkred,
    urlcolor   = darkred,
}


\usepackage{amssymb}
\usepackage{enumitem}
\setlist[itemize]{label=\textcolor{darkred}{\tiny $\blacksquare$}}
\renewcommand{\labelenumi}{\textcolor{darkred}{\arabic{enumi}.}}
\renewcommand{\labelenumii}{\textcolor{darkred}{\arabic{enumi}.}}
\renewcommand{\labelenumiii}{\textcolor{darkred}{\arabic{enumi}.}}

\begin{document}
\pagestyle{empty}

\par{\centering
		{\huge Antú G. Eyaralar}
	\bigskip\par}

\section*{\faAt ~~ Datos Personales} 
\hrule

\
\newline
\

\begin{tabular}{l | l}
      \faLinkedin ~ \href{https://www.linkedin.com/in/antú-eyaralar-409297221/}{LinkedIn} \
     & \faGithub ~ \href{https://github.com/Agelar01}{GitHub} 
\end{tabular}

\section*{\faBook ~~ Educación} 
\hrule

\
\newline
\

\begin{tabular}{l l}
    2020 - actualmente  & \textbf{Licenciatura en Ciencias de la Computación}\\
                    & Facultad de Ciencias Exactas y Naturales, Universidad de Buenos Aires \\ \\
\end{tabular}

\section*{\faFileCodeO ~~ Proyectos}
\hrule
\

\begin{itemize}
    \item Proyectos para \textit{Algoritmos y Estructuras de Datos} (Materia), con el objetivo de  implementar las estructuras de datos más comúnes y conocer sus complejidades (Java).
    \begin{itemize}
        \item\href{https://github.com/Agelar01/Algoritmos-y-estructuras-de-datos-/tree/main/Lista%20enlazada/main}{Lista doblemente enlazada}.
        \item \href{https://github.com/Agelar01/Algoritmos-y-estructuras-de-datos-/tree/main/ABB/main}{Arbol de Busqueda Binaria}.
       
    \end{itemize}
    \item Trabajos de \textit{Arquitectura Organización del Computador} (Materia),con el objetivo de comprender como los datos se ubican en memoria y el manejo de la misma (C).
    \begin{itemize}
        \item \href{https://github.com/Agelar01/Arquitectura-y-organizaci-n-del-Computador/tree/main/Programaci%C3%B3n%20orientada%20a%20datos%20y%20memoria%20din%C3%A1mica}{Programación orientada a Datos y memoria dinámica} 
    \end{itemize}
    \item Este \href{https://github.com/Agelar01/Resume}{currículum}, hecho en \LaTeX.
\end{itemize}


\
\newline
\


\section*{\faCogs ~~ Tecnologías y herramientas conocidas} 
\hrule
\

\
    
\begin{multicols}{4}
    \begin{tabular}{l}
        Python \\
        Git \\
        LaTeX
    \end{tabular}

    \begin{tabular}{l}
        ASM \\
        C
    \end{tabular}
    
    \begin{tabular}{l}
        Haskell \\
        Java
    \end{tabular}
    
    \begin{tabular}{l}
        JavaScript \\
        HTML/CSS  \\
        SQL
    \end{tabular}
   
\end{multicols}

\newpage

\section*{\faCoffee ~~ Skills} 
\hrule
\

\

\begin{multicols}{2}
    \begin{itemize}
        \item Autodidacta
        \item Lectura de documentación técnica
        \item Trabajo en Equipo
        \item Comunicación asertiva
        \item Resolución de problemas
    \end{itemize}
\end{multicols}

\section*{\faLanguage ~~ Idiomas}
\hrule

\
\newline
\

\begin{tabular}{l | l}
    Inglés escrito y hablado & Intermedio//Avanzado
\end{tabular}

\
\newline
\

\section*{\faHeartO ~~ Personal} 
\hrule

\
\newline
\

\noindent Me interesa aprender nuevas tecnologías y reforzar las conocidas no sólo para crecer profesionalmente, sino también para generar eventualmente un aporte al mundo.
\end{document}
